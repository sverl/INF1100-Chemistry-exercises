% !TeX root = exercises.tex
% !TeX encoding = UTF-8
% !TeX spellcheck = nb_NO

\chapter{Forord}
Dette heftet er skrevet som et tilbud til bachelorstudentene i kjemi 2015 og 2016. De tidligere kjemistuntene hadde emnet programmeringsemnet \emph{GEO--KJM1040 -- Grunn\-kurs i programmering for problemstillinger i geofag og i kjemi}, men ble erstattet med \emph{INF1100 -- Grunnkurs i programmering for naturvitenskapelige anvendelser} på grunn av at det tidligere emnet ikke ga de nødvendige programmeringsferdighetene som trengs senere i studiet. Siden \emph{INF1100} i utgangspunktet er beregnet for studenter på programmene \emph{Elektronikk og datateknologi}, \emph{Fysikk, astronomi og meteorologi}, \emph{Matematikk og økonomi}, \emph{Matematikk, informatikk og teknologi} og delvis \emph{Materialer, energi og nanoteknologi}, har oppgavene liten tilknytning til kjemi. 

Mange er oppgavene er basert på oppgaver fra et tilsvarende hefte for \emph{GEO--KJM1040}, \emph{GEO1040, Kjemirettede oppgaver} av Håkon Beckstrøm, Bård Andrè Bendiksen og Espen Hagen Blokkdal og noen er inspirert av oppgaver fra \emph{General Chemistry: The Essential Concepts} av Kenneth Goldsby og Raymond Chang. 

Mange av tallene i oppgavene er funnet på siden det er vanskelig å finne reelle tall på noen av oppgavene.

Har du spørsmål om oppgavene, forslag til forbedringer eller andre henvendelser til dette heftet, ta kontakt med Sverre Løyland på email, du finner den på  \href{www.uio.no}{uio.no}.